\documentclass[a4paper,twoside]{article}

\usepackage{float}
\usepackage{url}
\usepackage{epsfig}
\usepackage{subfigure}
\usepackage{calc}
\usepackage{amssymb}
\usepackage{amstext}
\usepackage{amsmath}
\usepackage{amsthm}
\usepackage{multicol}
\usepackage{pslatex}
\usepackage{apalike}
\usepackage{SCITEPRESS}     % Please add other packages that you may need BEFORE the SCITEPRESS.sty package.

\subfigtopskip=0pt
\subfigcapskip=0pt
\subfigbottomskip=0pt

\begin{document}

\title{Use of augmented reality to support education  \subtitle{Creating a mobile e-learning tool and using it with an inquiry-based aproach} }

\author{\authorname{Walter J. Rezende\sup{1}}
\affiliation{\sup{1}Institute of Informatics, UFG University, Goi\^ania, Brazil}
\email{\{Walter\}walterjrezende@gmail.com}
}

\keywords{\fontsize{9pt}{10.8 pt}\selectfont Augmented Reality, Education, Mobile Devices.}

\abstract{\fontsize{9pt}{10.8 pt}\selectfont Education is the base of human development, in latest years working professionals have demonstrated a considerable interest in new technologies searching to enhance teaching quality. Tools such as e-learning, cellphones, video-conferences, web quests and others are becoming popular options to help motivate and enrich the knowledge of students.  Now, with modern technology, the societies requirements for knowledge have gotten more specific and accumulated, however, the traditional teaching model hasn't been able to keep up. According to research from~\cite{moran2007educaccao}, what mainly pushes away students in their fifth to eighth school year in Brazil is lack of interest. Poor connection between life and subjects, added to the lack of connectivity between them and the virtual world reduces the chance to create good professionals and citizens. The development of tools to support teaching can help minimize these problems. AR is a promising technology for education. Together, with the help of proactive(inquiry-based) approach this will reduce content abstraction and increase motivation of students. This article demonstrates the construction of a mobile software educational tool using augmented reality(AR) technology and aiming to improve teaching experience. By presenting content of books in three dimensions, we creates an interactive and fun environment for children to learn.---resumir comeco, e completar o fim com o trabalho feito}

\onecolumn \maketitle \normalsize \vfill

\section{\uppercase{Introduction}}
\label{sec:introduction}

\noindent Augmented reality (AR) is one of the variations of virtual reality and can be seen as introduction of artificial stimuli over real ones, with the use of multisensory technology. In other words, we include virtual information on human senses, enhancing men-environment relationship.

The main characteristics present in AR are: mix of virtual elements in the real context; interactivity with resultant reality; positioning of virtual objects in a coherent manner according to the reality in question and influence over all senses~\cite{azuma2001recent}.

The accelerated changes in the technological context, have increased evaluation of teaching methods wold-wide~\cite{murnane2007problem}. As an outcome, we now need to find new ways to learn, a challenge for researchers and teachers searching for ways to maximize content absorption while preserving the individuality of thoughts.
	
According to Murnane, we now face a challenge of identifying, distributing and improving teachers for current demands.

Personal computers have revolutionized the way humans interact with the world. As a silent side-effect, the current on-screen metaphor inevitably holds the users attention, in order to manipulate its ``selfish" interface. By combining the best of both worlds, AR presents itself as an interesting solution for this problem and can be applied into almost any area, including education, that traditionally has resisted to significant changes~\cite{hannafin1993technology}.

Our proposal however, offers a different approach, not only suggesting the use of augmented reality for everyday class through mobile devices, but also by proposing to attach it to a proactive method. The implications of this in the student's role, changes it from passive learning to active, helping pupils develop self-teaching and collaborative work skills. ----EDITAR e revisar

The idea that will be presented in this paper, is to build a software tool that can offer students a different approach for learning. Combining both the AR book strategy and the inquiry-based learning model, we expect to reduce abstraction of subject content and offer a proactive and social experience that will help connect subjects to student lives. The choice of mobile devices offers us an economically viable way to execute this strategy and yet still maintain the quality of the experience, specially due to it's less exclusive interaction metaphor compared to common desktop applications.

\section{\uppercase{Motivation}}
\noindent AR Creates an intentional illusion that we can use to enrich the sense of vision. This allows to include information of the surroundings in our sight or of any other relevant data at the time, like schedules, notifications, updates, etc.  

Naturally this feature can be used in education and there are many projects involving it. The fact that we can mix reality with virtual objects is at least interesting to consider, imagine being able to show a model of a plant we study in biology or an object representing the area we are calculating in math for examples. 

Collaboration according to~\cite{kaufmann2003collaborative} is one of the most important parts of educational environment, yet still the traditional learning model rarely stimulates it, through group activities every now and then. Using a proactive collaborative model makes social activities common, and involves students in the process of teaching, allowing them to attempt their very own solutions yet still being aided and evaluated by supervisors.

The main advantages of AR include it's low requirements of hardware, as it is almost exclusively software dependent~\cite{zorzal2006realidade}. Frivolously speaking it requires not much more then a camera, processors, memory and energy source.

There are many available devices compatible with AR nowadays, this allows for a variety of strategies that make use of this approach, in this particular case, for educational purposes. 

Regular books present the content to pupils in a manner that requires imagination to fully understand what is being described. Augmented contexts on the other hand allows for us to reduce that abstraction, on replacing entirely or partially two dimension illustrations for three dimensional objects that could be interacted with.

The tool itself is part of the proposed solution for improving class deficiency, but it still requires getting adapted to the current educational context, and for that inquiry-based learning can help. Studies have shown improvement in classes with proactive methods, and we will get in to that later in this paper.

\subsection{Devices for application}
As we mentioned earlier, the hardware used for AR has very few restrictions, most of which are present in any computer based device. Now we will make a brief description of the main available devices, considering their limitations and advantages for this application:
\begin{enumerate}
\item Smartphones: Smartphones have become very popular in recent years, according to projections, more then 2 billion people will have smartphones by the end of 2016~\cite{2Billion75}. Another advantage is the mobility, allowing users to move freely while holding the device. The weight of a Smartphone is usually around a couple hundred of grams or less, very light compared most of the other devices.

Currently, most smartphones have processors capable complex image processing, and almost all of them have cameras and network connection of some sort. If we add their accessible price to the list, that makes them great candidates. On the other hand, the cons include a low battery life, specially while using Internet connection, and a large variation of camera configurations. These problems can be managed, as for the battery, most of them can last for hours, which is sufficient and even the worst cameras are still somewhat suitable for augmented reality purposes.

\item Tablets: According to projections of~\cite{Globalt81}, tablets have maintained approximately 200 million units shipped worldwide and should maintain less so for the next few years. That being said, this shows they have a good user base, while not nearly as large as the one on smartphones. 

As for the technical evaluation, they are nearly identical to smartphones, while having some particularities in their operating systems and some due to their larger size. On the pros, we have a large screen, allowing for a good observation and manipulation of the virtual objects. Another good feature is the possibility of better hardware, but that is not true in all cases, for there exists low configuration tablets and highly configured smartphones. As in phones, they may have mobile Internet connection, and if not, they can use wireless local Internet access. They are also heavier then smartphones, but still light enough to be handheld.

While being less popular, they are still nearly as suitable for AR, as smartphones. Another advantage is that they usually have systems compatible with common smartphone applications. We will not use them in this paper, but on future works support could be included.

\item Smart Glasses: A very modern device, that is worn as regular glasses but has augmented reality support embedded in their special lens. Empirically they seem like a great candidate, but the lack of studies around it added to their restricted access makes them unsuitable. Some companies have even advised for them not to be worn by children~\cite{FAQGoogl98}.

\item Laptops: The Portable computers or their most common name, laptops, are certainly an option. While they are portable, they are not handheld, and restricts movement. Like desktops they can have great hardware, large screens, and most of them have cameras, but the position faces the user which is not desirable. As for the price they are typically more expensive them mobile phones, and weight too much to move around, which is ideal for group activities and therefore will not be considered in this paper.
\end{enumerate}

\subsection{Contextual Learning}

\noindent According to~\cite{chiang2014augmented}, studies have shown the importance of context in learning, even stimulating the practice of outdoor teaching, with the aid of mobile and sensory technologies. 

The passive learning model, has survived countless generations and has functional results, but in its most traditional form, does not encourage discussions and imagination efficiently. On short, the traditional method consists of a teacher dictating the content followed by possible question, similar to lectures. Situation in which students tend to get anxious and unmotivated, specially for complex subjects and long classes. The inquiry-based teaching experience is very different, 
pupils have freedom to utilize their own sources and to choose their own methods, while working in groups, debating topics and finding information. This experience stimulates curiosity, social skills and searching, making them feel ``rewarded" upon realizing discoveries. At the same time, this method makes classes less dependent on the supervisor's teaching skills.

There are many ways to implement inquiry-based learning, centered in five general elements according to~\cite{4Phaseso86} citation of Indiana University Bloomington:
\begin{enumerate}
\item Learning focuses around a meaningful, ill-structured problem that demands consideration of diverse perspectives.

\item Academic content-learning occurs as a natural part of the process as students work towards finding solutions.

\item Learners, working collaboratively, assume an active role in the learning process.

\item Teachers provide learners with learning supports and rich multiple media sources of information to assist students in successfully finding solutions.

\item Learners share and defend solutions publicly in some manner.

AR can aid in this process, through the implementation of a solution that stimulates interaction with knowledge while searching for the solution. Today technology is becoming more ubiquitous and changing focus from constant computer and tool upgrades to influencing our decisions, our lives and connecting everyone to everything~\cite{galloway2004intimations}.

One of the best advantages of mobile devices is the possibility to take teaching to new places, innovating in paradigms that could prove to have better results for learning~\cite{fitzgerald2013augmented}. The author also mentions that mobile computing become a trend due to the Internet's ubiquity.
\end{enumerate}

\subsection{Objectives}
\noindent The proposition of this work is the elaboration of a system to support learning, manifested in a mobile application with multiple functionalities. This software will have 3 modules: reading of AR markers, activities and a game module that will be described in details later on this paper. The main goal of these modules can be summarized in the following manner:
\begin{enumerate}
\item Marker Reading: this module is responsible for the educational AR marker readings. These readings can be used whenever should the student be interested, and the details involving the AR reading will be explained later on. Summarizing the idea of this module, it works as follows: A object in 3 dimensions will projected from a valid marker, relative to the position and distortion of the pattern. This will create an illusion of the object that is related to the subject, in the current context, being able to be rotated for better observing with a touch on the screen. Another functionality will be a button that changes the projection into another one, valid and related to the same content, usually a different perspective, such as the inside of the object, or a slice, etc. To finish this modules description, the last functionality will be links suggested for learning about the topic, available when the object is projected through the scan of its marker.  

\item Activities: Upon being accessed, this particular module will show a list of groups that had been used and allow the user to add new ones. Upon selecting the desired group, a QR (Quick Response) should be scanned, and it will be handed to the students by the supervisor. After the scan, an activity will be loaded and the group will have the option to answer a multiple-choice quiz, about the content, to be done at the end of the activity, showing the result afterwards and sending relevant data to the teacher, for better understanding of the student's situation.

\item Games: The last module is for entertainment purposes, it will be a library of AR games that can be played when the students desire. The purpose of this module is to provide an educational and fun activity to help increase interest in the application and at the same time, get user familiar with using AR in this software. The details about this module will be described later in this paper, but it can be advanced that any content appropriate AR game would be suitable for use in future works, in particular those that can help develop cognitive thinking.
\end{enumerate}

\section{Technology and inquiry-based learning}
\noindent According to~\cite{kirkorian2008media} there is a positive and negative impact in children, with the use of electronic media, especially television. He also mentions that the largest impact of television, and probably of any type of media is related to the content more so then the time of use. This suggests that we should specially focus in choosing the correct content, that has to be adequate for the age of our niche.

Inquiry-based learning is a strategy that requires the posing of questions and scenarios to later connect them with facts or pieces of evidence, in order to reach conclusions. This method is important for the development of higher thinking skills, and can be supported with the aid of technology~\cite{edelson1999addressing}. As mentioned by the author, there is a trend in science, for using technology-supported, inquiry-based learning to allow new forms of inquiry, which is the foundation of science. He states that researchers identified six contributions technology can make for the learning process:
\begin{enumerate}
\item Enhancing interest and motivation.

\item Providing access to information.

\item Allowing active, manipulable representations.

\item Structuring the process with tactical and strategic support.

\item Diagnosing and correcting errors.

\item Managing complexity and aiding production.
\end{enumerate}

In this paper, our goal includes all these items listed above, in some level. By creating this software, it can be used to improve perception of knowledge, and at the same time, offer teachers a feedback. This can create a honest way to evaluate  student development, something that typically had to be done exclusively considering the tutors skills.

The main purpose of technology since its creation, is to support humanity in activities that sometime in the past were impossible or harder. This purpose helps maintain its constant evolution, and rarely do we judge the need, before it actually happens. It is known that practice is essential to maintain brain acuity, and sometimes technology can actually disturb this practice, although we use calculators for example, it is still essential to know how to calculate.

Following this reason, it is advised to consider the challenge when developing learning technologies for their purpose to begin with, is exactly to learn, and that includes not only the content, but the subtle skills required in the learning process that need to be developed. It is essential in inquiry-based learning due to its focus in problem solving, and therefore being able to fully understand the problem is ideal.

\subsection{Deploying technology based learning}
\noindent To help make decisions when implanting technology models, some works as~\cite{dahlstrom2013ecar} helps understand the challenges faced by institutions to initiate a project that attends to new demands, and the difficulty to preserve institutional identity. It is hard to choose the correct solution to improve learning, in an efficient and lasting manner. Considering that their niche is college students, it is still possible to learn from their experience, and two lessons are: students prefer to have a personal interaction with tutors despite all the technology resources offered, and the other is that they prefer to have access to sources at any moment.

According to~\cite{watson2013integrating} and the papers he cited, today the union of technology and learning became necessary. There are three steps to perform integration: integrate hardware and software, into disciplines and of teaching and learning. According to authors, this is a challenge for teachers. 

Another work that can help fixate the idea of the need to evolve current teaching methods, is illustrated in the work of~\cite{murnane2007problem}. As he mentions, the importance of cognitive skills for jobs. According to his numbers, the salary of low qualified employees in the United States have been stagnated or dropping due to globalized competition. Another fact listed is that cognitive skills serve as good measuring units, to evaluate educational accomplishment and are not being trained correctly in todays classes. This helps fixate the idea of the need for a change in our current educational context.
\section{Games and learning}
\noindent In recent years, researchers have seen the potential in games for educational purposes. According to~\cite{Cogni10} games are not bad for children, instead they are exactly the opposite, potentially beneficial and can help development of cognitive skills. He states that if they are of interest, they should be stimulated and can help improve attributes such as: perception, attention, memory and decision making. In another article~\cite{Video49} Gray states that video game addiction is a symptom, not a cause, and that it is related to general addiction problems.

Be as it may, technology is a reality, and therefore so are video games. We can use that to our advantage and adapt or stay rigid. As Albert Einstein once stated: ``The most beautiful thing we can experience is the mysterious. It is the source of all true art and science. He to whom the emotion is a stranger, who can no longer pause to wonder and stand wrapped in awe, is as good as dead"~\cite{calaprice2000expanded}. Considering options with an open mind is essential, and not every category of games is instructive~\cite{hogle1996considering}. According to him, the good games are fun, motivational and offer the right quantity of challenges, something that some classrooms already make use of.

\section{Augmented reality in education}
\noindent There are many researchers looking into the application of AR in education, using tools such as instructive games and markers books. Evolution in this niche can help make reality to a future that has only been pretentious in sci-fi movies up until now. In these films people would interact with their context through movements and virtual objects manipulation. 

Its very intuitive to imagine the possible gains from the use of AR in classrooms. 

Today students are limited mostly to imagine the content they hope to learn from books and other sources, digital or physical, an easy task for some, not as much for others. With the use o AR, we can change this reality, by using virtual models to simulate a convincing, almost real, display of the content. This exposes the object of interest in a three dimensional interactive form, more natural then its current representations, and one day can become ubiquitous, such as image projectors and the Internet have become~\cite{cardoso2014uso}.

In the software that will be developed in this paper, an alternative that includes social interaction is considered. The need to discourage methods that individualize learning, such as using individual computers is a reality~\cite{billinghurst2002augmented}. According to his paper, even when neighbors, this methodology reduces effectiveness of pupils, and very commonly they feel this and end up grouping spontaneously.

To solve this problem, we will use a method that includes dividing resources, in this case the mobile devices. Such practices can help to understand the need for sharing, and how that can help increase productivity, while relying on others to get the job done. For more information on this, it is advised to look up at~\cite{svanaes2000search} work. For this paper we will be including the Jigsaw method that will be explained shortly in the next section.

Before we start describing the Jigsaw method, we should look at~\cite{wu2013current}. He classifies three major categories for approaches that use AR in education, its fair to note that there are many possibilities for mobile devices in education, and that they adapt well to AR technologies, these categories are:
\begin{enumerate}
\item Emphasizing the roles: ``Approaches emphasizing engaging learners into different roles in an AR environment included participatory simulations, role playing, and jigsaw approach. Because these approaches emphasize the interactions and collaboration among students, they are usually associated with mobile-AR, multiplayer AR, or game-based AR. Participatory simulations can be defined as allowing “different players to function as interacting components of a dynamic system” and consequently interactions among students affect the outcomes of the system".
\item Emphasizing the locations: ``Place-based or location-based learning emphasizes learners’ interactions with the physical environment so mobile-AR with locationregistered technology is a common subset used for this approach. AR environments that take these approaches exploit the advantages of mobile technologies because mobile devices make it possible for computer servers to track learners’ actual geological location".
\item Emphasizing the tasks: ``The third category is centered on the design of learning tasks in AR environments. The approaches that can be identified in this category are: game-based, problem-based, and studio-based learning. Because of the diverse nature of the tasks, the implementation of these approaches does not necessarily rely on a specific subset of AR technologies. Among the approaches, game-based learning is one of the most popular for AR. AR games can be defined as `games played in the real world with the support of digital devices that create a fictional layer on top of the real world context'".
\end{enumerate}
\section{The Jigsaw method}
\noindent The Jigsaw method is a technique that works as a game, hence the origin of it's name, from the Jigsaw Puzzle. This method consists in partitioning the problem into smaller problems and then divide them between members of the same group, that later will explain to the other members his research results on the topic. 

In this approach, students become at the same time, learners and teachers, on a collective environment.

There are many studies that proof the efficiency of the Jigsaw method. One of these studies is the work of~\cite{kilic2008effects}, where an experiment was realized involving two groups, with nearly the same scores on a quiz handed out beforehand, and that then were exposed, the first to the traditional method, and the second to the Jigsaw approach. These groups then were tested once more in a new quiz after the lessons, on the results, the group exposed to the new method, scored 92,25 out of 10,00 while the other group, exposed to the traditional method, scored 75,5.

With AR as a tool, we will provide additional help, allowing members of these groups to experience the simulation of content we described earlier in this paper. With different subtopics handed out to each member, they could look at the same model, and therefore extract different information related to their particular focus and interest.

According to~\cite{TheJi25} definition of the Jigsaw method, we summarized the definitions into three steps, as follows:
\begin{enumerate}
\item Dividing in groups: The pupils are divided in small groups containing from five to six students, and an activity defined by the advisor for them to divide in subtopics between themselves.
\item Experts reunion: Each member with their own subtopic then researchers about it until the familiarize, then, each specialist from different groups with the same subtopic, gather to discuss it and reach conclusions.
\item Group Reunion: After the last step, the original groups gather once more, each subtopic being explained to the group by its respective expert. In the end, give a quiz to check the results of each group. 

During test of~\cite{kilic2008effects}, each pupil had a week to research about their own topic, before needing to explain it to their group. We need to consider the size of the task, in order to adapt this time to realistic measures, and to preserve the quality of the class.
\end{enumerate}
Finally, this method can be of great help, allowing and AR mobile tool to extract more of its potential. According to~\cite{milhomemabordagem}, a technology that aims to give digital support to real experiences, are best used if the actual experiences are still stimulated.

\section{Design patterns}
Applications for using AR can be very diverse, therefore some design patterns were developed to simplify the implementation in these systems. The work of~\cite{InsideOu42} states that Robert Rice, CEO of Neogence identified 3 recurring problems in AR applications for browsers, one is that they all try to make applications for their browsers instead of sharing, another is that different applications cant share resources, the last is the elaboration for single users. The author proposes 4 patterns to resolve these issues:
\begin{enumerate}
\item Heads-Up Display: ``The Head-Up Display interaction pattern echoes the targeting and navigation displays in military and other aircraft ... This is the oldest of the AR interaction patterns. Augmented experiences using the Head-Up Display pattern add information about the real objects in view into a complete mixed-reality experience that built-in AR tools and devices generate. While many AR experiences rely on external devices—that is, external to the body—those using the Head-Up Display pattern commonly depend on hardware that is integral to a vehicle or cockpit-like physical setting".
\item Tricoder: ``The Tricorder interaction pattern became familiar to us in the original Star Trek TV series. Mr. Spock, in a landing party on the surface of some new planet, would use his Tricorder scanning device to explore the local area ... Most of the time, interactions involved waving the Tricorder around in the air in the general direction of interest, while peering at the display screen. This is the dominant pattern of physical behavior when using AR browsers such as Wikitude demand. The essence of the Tricorder interaction pattern is that it adds pieces of information to an existing real-world experience, representing them directly within the combined, augmented-reality, or mixed-reality experience".
\item Holoches: ``Named after the circular, chess-style game Chewbacca and R2-D2 played aboard the Millenium Falcon in Star Wars, ... the Holochess interaction pattern adds new and wholly virtual objects directly into the augmented experience, combining them with existing, real objects. The virtual items in Holochess interaction patterns often interact with one another—and sometimes with the real elements of the mixed-reality experience."
\item X-Ray Vision: ``Superman used his powers of X-ray vision to look through walls and see concealed weapons—all in the name of stopping evil—as shown in Figure 13. In augmented reality, the X-ray Vision interaction pattern simulates seeing beneath the surface of objects, people, or places, showing their internal structure or contents. AR experiences using the X-ray Vision pattern often use a combination of projection and rendering—frequently, a schematic or abstracted rendering—of the object of interest, as in Medical Augmented Reality (MAR)."
\end{enumerate}

For the scope of this software, we will make use of the Holochess pattern, and more information will be described later in this paper.
\section{Related works}

The heading of a section title should be in all-capitals.

Example: \textit{$\backslash$section\{FIRST TITLE\}}
\noindent 

\subsubsection{Sub-Subsection Titles}

The heading of a sub subsection title should be with initial letters
capitalized (titlecased).

Words like "is", "or", "then", etc should not be capitalized unless
they are the first word of the sub subsection title.

Example: \textit{$\backslash$subsubsection\{First Subsubtitle\}}

\subsubsection{Tables}

Tables must appear inside the designated margins or they may span
the two columns.

Tables in two columns must be positioned at the top or bottom of the
page within the given margins. To span a table in two columns please add an asterisk (*) to the table \textit{begin} and \textit{end} command.

Example: \textit{$\backslash$begin\{table*\}}

\hspace*{1.5cm}\textit{$\backslash$end\{table*\}}\\

Tables should be centered and should always have a caption
positioned above it. The font size to use is 9-point. No bold or
italic font style should be used.

The final sentence of a caption should end with a period.

\begin{table}[h]
\caption{This caption has one line so it is
centered.}\label{tab:example1} \centering
\begin{tabular}{|c|c|}
  \hline
  Example column 1 & Example column 2 \\
  \hline
  Example text 1 & Example text 2 \\
  \hline
\end{tabular}
\end{table}

\begin{table}[h]
\caption{This caption has more than one line so it has to be
justified.}\label{tab:example2} \centering
\begin{tabular}{|c|c|}
  \hline
  Example column 1 & Example column 2 \\
  \hline
  Example text 1 & Example text 2 \\
  \hline
\end{tabular}
\end{table}

Please note that the word "Table" is spelled out.


\subsubsection{Figures}

Please produce your figures electronically, and integrate them into
your document and zip file.

Check that in line drawings, lines are not interrupted and have a
constant width. Grids and details within the figures must be clearly
readable and may not be written one on top of the other.

Figure resolution should be at least 300 dpi.

Figures must appear inside the designated margins or they may span
the two columns.

Figures in two columns must be positioned at the top or bottom of
the page within the given margins. To span a figure in two columns please add an asterisk (*) to the figure \textit{begin} and \textit{end} command.

Example: \textit{$\backslash$begin\{figure*\}}

\hspace*{1.5cm}\textit{$\backslash$end\{figure*\}}

Figures should be centered and should always have a caption
positioned under it. The font size to use is 9-point. No bold or
italic font style should be used.

\begin{figure}[!ht]
  %\vspace{-0.2cm}
  \centering
   {\epsfig{file = SCITEPRESS.eps, width = 5.5cm}}
  \caption{This caption has one line so it is centered.}
  \label{fig:example1}
 \end{figure}

\begin{figure}[!ht]
  \vspace{-0.2cm}
  \centering
   {\epsfig{file = SCITEPRESS.eps, width = 5.5cm}}
  \caption{This caption has more than one line so it has to be justified.}
  \label{fig:example2}
  \vspace{-0.1cm}
\end{figure}

The final sentence of a caption should end with a period.



Please note that the word "Figure" is spelled out.

\subsubsection{Equations}

Equations should be placed on a separate line, numbered and
centered.\\The numbers accorded to equations should appear in
consecutive order inside each section or within the contribution,
with the number enclosed in brackets and justified to the right,
starting with the number 1.

Example:

\begin{equation}\label{eq1}
    a=b+c
\end{equation}

\subsubsection{Program Code}\label{subsubsec:program_code}

Program listing or program commands in text should be set in
typewriter form such as Courier New.

Example of a Computer Program in Pascal:

\begin{small}
\begin{verbatim}
 Begin
     Writeln('Hello World!!');
 End.
\end{verbatim}
\end{small}


The text must be aligned to the left and in 9-point type.

\vfill
\subsubsection{Reference Text and Citations}

References and citations should follow the Harvard (Author, date)
System Convention (see the References section in the compiled
manuscript). As example you may consider the citation
\cite{Smith98}. Besides that, all references should be cited in the
text. No numbers with or without brackets should be used to list the
references.

References should be set to 9-point. Citations should be 10-point
font size.

You may check the structure of "example.bib" before constructing the
references.

For more instructions about the references and citations usage
please see the appropriate link at the conference website.

\section{\uppercase{Copyright Form}}

\noindent For the mutual benefit and protection of Authors and
Publishers, it is necessary that Authors provide formal written
Consent to Publish and Transfer of Copyright before publication of
the Book. The signed Consent ensures that the publisher has the
Author's authorization to publish the Contribution.

The copyright form is located on the authors' reserved area.

The form should be completed and signed by one author on
behalf of all the other authors.

\section{\uppercase{Conclusions}}
\label{sec:conclusion}

\noindent Please note that ONLY the files required to compile your paper should be submitted. Previous versions or examples MUST be removed from the compilation directory before submission.

We hope you find the information in this template useful in the preparation of your submission.

\section*{\uppercase{Acknowledgements}}

\noindent If any, should be placed before the references section
without numbering. To do so please use the following command:
\textit{$\backslash$section*\{ACKNOWLEDGEMENTS\}}


\vfill
\bibliographystyle{apalike}
{\small
\bibliography{example}}


\section*{\uppercase{Appendix}}

\noindent If any, the appendix should appear directly after the
references without numbering, and not on a new page. To do so please use the following command:
\textit{$\backslash$section*\{APPENDIX\}}

\vfill
\end{document}

